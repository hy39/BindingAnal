\documentclass[12pt,a4paper]{article}

% Always needed
\usepackage{times}
\usepackage{parskip}
\usepackage{graphicx}
\usepackage{setspace}
\usepackage{url}
\usepackage{lscape}
\usepackage{multirow}


% Optional
% \usepackage{lscape}
% \usepackage{multirow}
% \topmargin 0.0cm
% \oddsidemargin 0.2cm
% \textwidth 16cm
% \textheight 21cm
% \footskip 1.0cm

%\onehalfspacing
\doublespacing
\raggedright
\begin{document}
Classification: 

\large {\bf Hemagglutinin Receptor Binding Avidity Drive Influenza Antigenic Drift through the combination of both deleterious and beneficial Effects} \normalsize

Hsiang-Yu Yuan$^{1}$, James Hay$^{1}$, Katia Koelle$^{2}$

\begin{tabbing}
$^1$ 	\= MRC Centre for Outbreak Analysis and Disease Modelling \\
		\> Department of Infectious Disease Epidemiology \\
		\> School of Public Health \\
		\> Imperial College London \\
		\> London, United Kingdom \\ \\
		
$^2$ 	\> Department of Biology \\
        \> Duke University \\
        \> Durham, NC United States \\ \\
		

\end{tabbing}
\doublespacing
$^*$Corresponding author: hyuan@imperial.ac.uk, katia.koelle@duke.edu 
\clearpage


{\bf Results}

% The E of Binding Avidity without Drift 


% The Benefitial & Deleterial Effect of Binding Avidity Adaptation  
 
Binding avidity prolonged the survival of the viral lineages but reduce the maximum incidence. To understand the impact of binding avidity on disease transmissibility, the transient disease dynamics were first constructed without no antigenic drift. The virus binding avidity of the seeding virus is set to be optimum for the initial immune status. Comparing to the viruses without changes of binding avidity, the adaptation of binding avidity prolonged the viruses incidences (Fig4A), however, reduce the overall peak incidence.


During the outbreak, the amount of naive individuals (antibody level J=0) decreased contributing to the increased of higher immunity individuals in the population (Fig4B). The immunity increased rapidly while the incidence was approaching or near the maximum peak due to that more individuals' immunities were boosted. Later the increases of the immunity became slower after the peak until disease faded out. Virus binding avidity started to increase to adapt to the immunity when the naive individuals dropped and host immunity became higher during 200-400 days(Fig5). The average adaptation of binding avidity was determined by the viral fitness, which was measured by the reproductive number R0 assuming all the temporary recovered became susceptible in a total susceptible population. 


After the peak of the outbreak, the average binding avidity adapted along the maximum fitness overtime to prolong the viruses lineages, however, decrease of peak incidence was observed. Individually, if infection occurred among naive or more partial protection individuals, viruses could adapt to either too low or too high binding avidity within the hosts and would there decrease the fitness and produce deleterious effects for disease transmission (Fig2). The average final binding avidity was deviated to optimum virus binding was caused by the short-sighted evolution  when lots of naive individuals were infected.

When the viral phylogeny was reproduced, stabilised selection was observed for internal virus further demonstrate the deleterious effects of binding avidity  (Fig6), which is consistent to the higher proportion of netcharge charge in tips than in internal nodes in observed phylogenetic tree(previous figure). To verify the cause of the difference between internal and external strains, we created a model to represent serial passaging mechanism... 

 

   
% The Effect of Binding Avidity with Drift
To understand long term effects how binding avidity affect the antigenic drift in Influenza history, antigenic mutations were included. The additional antigenic changes caused by the binding avidity changes were included. We assume only weak antigenic changes as the by-product of binding avidity changes. At this case, binding avidity can increase the drift during the time when viruses keep spreading. However, if the antigenic changes are large, the host select the antigenic mutants mainly because of the fitness change on antigenicity and the binding avidity changes become compensatory mutations that correct the loss of fitness.

When only random antigenic mutations are included, disease incidence persist and oscillate occur mainly due to the transient effects of the recovered individuals. Lineage turnover events are (or are not) obvious.  When Binding avidity are included, after xx years, turnover occurred. 
1. Show incidence.
2. Show phylogenetic tree.
3. Show Binding avidity change and Sk distribution.
4. Show Antigenic mutation changes.

%The benefitial effects of compensatory mutation
5. Define Sk for low immunity, high immunity.
The selection coefficient of the antigenic changes were calculated with population under low, medium and high immunity. Without compensatory mutations on binding avidity, selection coefficient almost increase linearly with antigenic distance and reached the saturated levels. Higher immunity allows a higher level of maximum selection coefficient. fitness became limited because all the individuals are already naive. Furthermore antigenic changes won't have more contribution to infection. If the compensatory mutations occurred accompanied with antigenic changes, the selection coefficient increase almost 50 times higher. Furthermore, when the population immunity increased, the binding avidity adaptation reduce the effect of little antigenic mutations but enhance the effect of large mutations. 




    

 

{\bf Material \& Methods}

\textit{Analysis of HA cellular receptor binding avidity} \\
The RDE binding avidity for each single amino acid changes were both collected 34 mutants from the publicly available literatures (\cite{Hensley2009} \cite{Das2011} \cite{Myers2013} \cite{Li}) and given by the group performing the RDE assay for monoclonal antibodies selected sites (\cite{Hensley2009}). Totally there were 74 mutants collected, including 53 single, 20 double and 1 triple amino acid changes. For the RDE binding avidity assay present in literatures, the magnitudes were extracted using the Web Plot Digitizer(\cite{Rohatgi2012}). The effects of each single amino acid changes on binding avidity were calculated as the log ratio of RDE activity of the mutant to the wild type (TableS2).

\textit{Calculating net charge of influenza HA} \\
The influenza virus sequence isolated from New York State as part of an Influenza Genome Sequencing Project (\cite{Ghedin2005}) along with the exact date of isolation and clinical metadata on host age were collected from the Influenza Virus Resource Database (\cite{Bao2008}). We obtained in total 686 full-length influenza A/H3N2 HA sequences isolated between 1993 and 2005. Changes of the net charge were estimated by summing the total number of positive and subtracting the negative  charged amino acids. Amino acid such as Arginine (R), Histidine (H) and Lysine (K) are positively charged while Aspartic acid (E) and Glutamic acid (D) are negatively charged. For estimating net charge from the isolated viral HA sequences, net charge was calculated from the sequences using amino acid positions 17 through 345 of the HA, which defined the protein’s globular head domain (HA1). We specifically did not calculate net charge values using only the amino acid positions that comprise the cellular receptor binding site because charge at other sites in the globular head domain also impact electrostatic forces and thereby binding avidity. The calculation involved subtracting the total number of negatively charged amino acid residues from the total number of positively charged amino acid residues.

\textit{Phylogenetic Analysis} \\
Phylogenetic trees were reconstructed using the software program BEAST (\cite{Drummond2012}), under a general time reversible (GTR) model with gamma distributed rate variation and a proportion of invariant sites. 107 MCMC steps were sufficient in reaching a high effective sample size. From each viral subtype’s maximum clade credibility (MCC) tree, we inferred the ancestral states of the observed viral sequences. If BEAST produced multiple possible ancestor sequences, consensus sequences were calculated with BLOSUM50 scoring matrix using Matlab Bioinformatics Toolbox. For all the observed external and inferred internal sequences we calculated net charge as described above.


\textit{The Individual based modelling} \\
We created an individual based model, in which each individual host was represented as the statuses: susceptible ($S$), infected ($I$) and recovered ($R$), respectively with the finely scaled life course immunity $K$ (FigureS4A). The life course immunity was defined as the immunity against the first infected strain or other strains with the same antigenicity. We consider each virus as an unique viral strain replicating in the infected host and possessing certain phenotypic characteristics such as the antigenic change $\delta$ and the binding avidity $V$. The infection histories $h$ of the hosts were able to be tracked by specifying the set of viral strains to which each individual host was exposed to. Since each viral strain underwent antigenicity changes, we also defined a pairwise immunity $J$ as the host immunity, given the infection history, against the challenging virus $v$.

The model was simulated under demographic stochasticity using Gillespie’s tau-leap algorithm (\cite{Gillespie2001}) to calculate the probability of number of events for each individual in an unit time interval. We assumed that births of new hosts, starting from no life course immunity ($K = 0$), occurred at a rate 1/70 per year. Similarly, deaths occurred also at the same rate for the entire population. Total number of contacts was calculated based on the contact rate c, number of susceptibles X, and number of infecteds Y in a total population N, resulting in the total number of contacts cXY/N in a unit time. As shown in FigureS4A and B, for each contact, a random susceptible target $S_{tar}$ and an infected source $I_{src}$ were drawn from the population, and the probability of successful transmission, $\rho$, was determined by the pairwise immunity $J$ of the target host against the challenging source virus $v_{i}$, and the binding avidity V of the virus at the contact time (FigureS4B). The pairwise immunity , $J =K - min(\delta_{ih})$, was the life course immunity that subtracted the minimum antigenic distance between the current challenging virus $v_{i}$ and the previous infected strains in the infection history. Once the individual became infected, the average infectious period $1/\gamma$ would lasted 3.3 days, during which the virus’ binding avidity $V$ was adapting to the host immunity $J$ determined by the fitness gradient (\cite{Yuan2013}) and producing the antigenic changes $\delta$.

After the infected individual was recovered, the individual became transiently fully protected within 25 days ($1/\omega$) on average by short-lived immunity, possibly mediated by cytotoxic T lymphocytes with other immunological factors ( [39] [40]). During the recovered period, individual life course immunity $k$ was boosted following a Poisson distribution with mean boosting level $g=3$, which produce nearly $70\%$ clinical protection during primary infection, similar to our previous estimation (\cite{Yuan2016}). After the transient immunity waned, the recovered individual became susceptible again and were protected from viral infection by the increased antibodies pairwise immunity $j$.


\textit{Probability of infection} \\
Each time when a virus $v$ was contacted to a susceptible host $S$, the probability of successfully infecting that host was defined as the likelihood of the viruses not going stochastic extinction in a viral population within that host:

$\rho = \frac{1}{1-(R_{in})^\sigma}$

where $R_{in}$ is the reproductive number in a host and $\sigma$ is the number of virions initially transmitted. $R_{in}$ was defined as the expected number of viruses which can be produced by a single virus in a host with immunity J.

$R_{in}=n \cdot f(J,V) \cdot g(V)$



Where $f(J,V) = [1 - e^{p(V+1)}]^r$ and $J$ is the probability of evading the immune response with a given virus binding avidity $V$, and a given host immunity $J$ ;  and $p$ and $r$ are scaling constants. $g(V) = e^{(aV^b)}$is the probability of successful replication within a host with a given binding avidity $V$, where $a$ and $b$ are scaling constants (\cite{Yuan2013}). Combining these probabilities with the average number of offspring $n$ produced after the replication by a single virus give the reproductive number for viruses within a host.


\textit{Pairwise immunity and immune history} \\
We considered host pairwise immunity as having partial protection against different viruses depending on the nearest antigenic distance between the infecting virus and the viruses within the host’s immune history. The pairwise immunity of the target host $t$ against the challenging virus $v_{i}$ in the infected source, was given by $J_{mi}=K_{m}-min(D_{i\hbar_{m}})$, where $K_{m}$ is the life course immunity of host $m$ against the primary infecting strain or the same strains without antigenicity changes, $D_{i\hbar_{m}}$ is the set of the antigenic distances between the infecting virus i and the set of infected viruses among infection history $\hbar$ in the host $m$. Following a contact, we find the shortest antigenic distance $min(D_{i\hbar_{m}})$ between the current challenging virus and the previous viral strains in host’s infection history by tracing back along the transmission tree to find the last common ancestor of the infecting virus $v_{i}$ and the infection history viruses $\hbar_{m}$ and calculate the sum of each individual antigenic changes along the path.


\textit{Within host adaptation and antigenicity change} \\
During infectious period, binding avidity $V$ of each infected strain adapted to the host immunity $J$. The change of binding avidity by time $\Delta V(t)$ is the product of the rate of change of the viral binding avidity $dv/dt =V_{c}(dR_{in}/dv)$ and the time elapse $\Delta t$ (\cite{Yuan2013}), where $V$ is the within-host binding avidity of the virus, $dR_{in}/dV$ is the fitness gradient, and $V_{c}=0.075$. We considered the scenario where antigenic drift occurred as sum of the epitope changes that contains both randomly produced antigenic mutations, which comply with the classical hypothesis of antibody selected antigenicity changes, and the mutations produced as the by products of binding avidity changes. Random antigenic mutations occurred with a probability $p$, and the effect of the antigenic change $r$ was drawn from an exponential distribution with average antigenic distance $d=0.1$ could occur. Changes in binding avidity also translated into small changes in antigenicity described by parameter $\kappa$, such that rate of antigenic change due to binding avidity change was proportional to the rate of change of binding avidity. Therefore the total antigenic change happened in one virus particle was expected to be

$\Delta A=rp+\kappa \Delta V.$


\textit{Initial population immunity} \\
We began each simulation using initial conditions of the immune profile of susceptible individuals that generated from a single epidemic peak. To match the distribution of partial immunity that were observed in population, we ran the simulation from a completely naive population to generate a distribution of host immunities. The epidemics generated a total incidence of roughly 50\%. Each infected individual received a boost to immunity drawn from a Poisson distribution with mean $mu=6$, resulting in a distribution of host immunities, $K$. All simulations were then run using the same host immunity distribution, $S_{k}$, introducing a new virus with antigenic distance $\Delta A=2$ from the host’s immunity. The resulting distribution of $J$ was overall similar to serological data for seasonal influenza (\cite{Yuan2016}).


\textit{The fitness in a changing immune profile environment} \\
The effective reproductive number was calculated while the total number of and the immune profile composition of susceptible individuals changed by time. Given a single virus with binding avidity V, the reproductive number was:

\begin{equation}
R=\frac{\sum_{J=0}^{\infty}\beta(J,V)S_{t}(J)}{\gamma N}
\end{equation}

The average within host relative fitness $\bar{w}$ was calculated from the the immune profile composition of susceptible individuals. We first calculated the newly infected individuals with different immunity, because all the mutants are replicated only in the infected hosts. The number of newly infected individuals is
\begin{equation}
I_{tn}(J) = \beta(J,V)\frac{S_{t}(J)}{N} {I_{t}}
\end{equation}
Under the assumption that the fixation of the mutation is proportional to its within-host fitness in infected individuals (\cite{Gillespie1984}). The the average within host relative fitness $w$ is
\begin{equation}
\bar{w}_{t}(V)=\frac{\sum_{J=0}^{\infty} R_{in}(J,V)I_{tn}^*(J,V)}{\max{R_{in}(J,V)}}
\end{equation}
where $I_{tn}^*(J,V)$ is normalized to be $\frac{I_{tn}(J,V)}{\sum_{J=0}^{\infty}I_{tn}(J,V)}$ to consider the scenario that every initial binding avidity are equally distributed. 

\clearpage

{\bf Figure Legends} \\
Figure1. The impact on binding avidity of single amino acid mutations by net charge. The value is calculated as the log ratio of RDE activity of the mutant to the wild type. Positive represents the amino acid change that increase net charge by 1 or 2 units. Negative represents the amino acid change that decrease net charge by 1 or 2 units. Neutral represents net charge is same. Charged amino acids include aspartate (D), glutamate (E) are counted as negative (-1) and Histidine (H), Arginine (R), Lysine (K) for positive (+1) are counted as positive (+1).
      

Figure2. Net charge distribution by age group. The proportion of higher net charge is calculated for 5 age groups. The viruses sequences and clinical data are obtained from the Influenza Genome Sequencing Project (\cite{Ghedin2005}) and the Influenza Virus Resource Database (\cite{Bao2008}).  

Figure3. Within host fitness and the probability of infection. (A) Within-host fitness represented by the expected reproductive number $R_{in}$. Parameters for within-host immune escape and replication cost, p=4, r=70, a=0.7 and b=3, and parameter for virus offspring number $n=4$. Bluer colours indicate much naive individuals where redder colours indicate much immune individuals. (B) The probability of infection with differential selection. Within-host parameters values are same as (A) and the effective initial number of the transmitted virus $\sigma$=1. (C) The probability of infection with weak differential selection. Parameters values same as (A) but the effective initial number of the transmitted virus v=3 to resembles higher viral dosages in serial passaging transmission. 

Figure4. Incidence over time for the various binding avidity scenarios. The blue line shows incidence when binding avidity is fixed near the late stage population average of V=0.6, eliciting a rapid, large peak. The green line shows the incidence when binding avidity is fixed near the early phase population average at V=0.45, resulting in a very small epidemic peak. The red line shows the incidence when binding avidity is allowed to adapt within hosts over the course of the epidemic, starting at a binding avidity of V=0.45. When binding avidity is allowed to adapt within the host, the overall incidence is higher than in the suboptimal binding avidity case, but lower than the scenario with fixed late-phase average binding avidity. In all scenarios, Total population N is set to be 1,000,000. For the initial day, the number of seed viruses I0 = 100, recovered hosts R0 = 0 and the remaining hosts were susceptible under the initial immunity distribution as described in Methods. Contact rate, c=0.7, birth/death rate, m=1/(70*365)(days-1), waning rate, w = 1/25, recovery rate, g = 1/3.3 (days-1), genetic variance, Vc=0.075, mean boost to immunity, mu = 6, initial antigenic distance, delta=2. Within host parameters were as described in FigureS1. Initial conditions were generated as described in the Methods.

Figure5. Distribution of susceptible host immunity over time for the adaptive binding avidity scenario as described in Figure4. Colours show the proportion of susceptible hosts with a given level of effective immunity against the seed virus, J, over time. The white line represents the population mean immunity level. At the start of the simulation, the majority of hosts are completely naive to the novel strain, with some hosts exhibiting low levels of immunity due to prior infection. As incidence increases, the mean host immunity also increases as infected hosts recover and develop long-term immunity. This coincides with the peak of incidence at around 200 days. Parameters are as described in Figure4 and FigureS1.

Figure6. The effective reproductive number changes by time and binding avidity. The effective reproductive number was calculated assuming the population contains only susceptible individuals with different protection (as Figure5) but excluding the transient recovered individuals and infected individuals. The optimum binding (dotted line) produces the largest number of offsprings. Red, the mean starting binding avidity during infectious period. Light blue, the mean final binding avidity during infectious period. Bold gray, the optimum reproductive number. Thin gray, the optim reproductive number when binding avidity is 0.

Figure7. Within-host and between-host relative fitness by different binding avidity and antigenic  changes. Within-host relative fitness is shown in red surface. Between-host relative fitness is shown in a surface from blue (low) to yellow (high).

Figure8. Binding avidity changes from the sampled viral phylogeny. Blue dots, internal nodes from the viral phylogeny. Green dots, external tips from the viral phylogeny. Dotted line was the optimum virus binding avidity that will produce largest offsprings. Bold gray, mean binding avidity from internal nodes. 

Figure9. Cellular receptor binding avidity dynamics analyzed phylogenetically. (A) The maximum clade credibility (MCC) tree for the H3N2 viral isolates from New York State, with HA net charge values mapped onto all inferred internal nodes and all observed external nodes. Numbers next to the bracketed clades denote the $\#NGS$ group. (B) Viral net charge dynamics over time. Each point represents a node from (A) and is colored by whether it is classified as an internal trunk node (red circle), an internal non-trunk node (blue dot), or an external node (green ‘x’). Its placement along the x-axis corresponds to the inferred or observed time of the node in the phylogeny. Its placement along the y-axis corresponds to its calculated net charge value. Each of the four subplots shows viral net charge dynamics for a single $\#NGS$ group. The total number of net charge changes occurring along internal branches (including trunk and non-trunk internal branches) was approximately $40\%$ less than the total number of net charge changes occurring along external branches. Normalizing by total branch lengths of internal and external branches yielded similar results, with fewer number of net charge changes occurring per unit time on internal relative to external branches.



\clearpage
{\bf Supplementary} 
\clearpage

\begin{table}
{\bf Table S1.} Comparison of the net charge properties in both internal and external nodes from the reconstructed phylogenetic tree.

\begin{tabular*}{10cm}{rrrrrr}
\hline\hline \\%inserts double horizontal lines


Net charge&  Dec. or Neu. &  Inc. & Total  & \\
\hline \\ %inserts double horizontal lines
$ \geq 0$ &  39           &  10   & 49     &  \\
$ < 0$ &   6           &  19   & 25     &  \\
Total  &  45           &  29   & 74   &  \\
\hline %inserts double horizontal lines
\end{tabular*}
\end{table}
P value = 0.0001 using Fisher's exact test

\clearpage


\begin{table}
{\bf Table S2.} The comparison of the seroprevalence observed sera and the titre model output. The seropravalene (\%) among each age group during the baseline and the follow-up rounds are listed.

\begin{tabular*}{16cm}{rrrrrrr}

\hline\hline \\%inserts double horizontal lines

   & Total strains&  \parbox[c]{1.8 cm}{\raggedright No. strains change netcharge}  &    \parbox[c]{1.8cm}{\raggedright Positive changes} &   \parbox[c]{1.8cm}{\raggedright Negative changes} &     \parbox[c]{1.8cm}{\raggedright Prop High Netcharge}    &     \parbox[c]{1.8cm}{\raggedright Avg netcharge}   \\
\hline \\ %inserts double horizontal lines

Internal &  684 & 53 (7.7\%) &  15 &   38 & 20.0 (\%) & 17.52 $\pm 1.13$\\ \\

External &  686 & 83 (12.1\%) &  32 &   51 & 18.7 (\%)& 17.45 $\pm 1.18$\\ \\


\hline %inserts double horizontal lines
\end{tabular*}
\end{table}

\clearpage


{\bf Table S3.} The list of all the collected mutations with the corresponding binding avidity (RDE). (Attached as the excel file)


\clearpage


FigureS1. The relative binding avidity by absolute net charge. The value is calculated as the log ratio of RDE activity of the mutant to the wild type. Absolute net charges of all the mutant strains are calculated. Regression is performed. P-value = x.xx. 

FigureS2. Comparison of within-host and between-relative fitness by time during the epidemics.
Red line represents the average relative fitness within newly infected hosts. Blue line represents effective reproductive number in the human population.

FigureS3. Viral phylogeny with weak differential selection on binding avidity. Parameters setting are same as Figure S2A with contact rate c=0.3. 

FigureS4. Schema of the disease transmission in the individual-based model. (A) The statuses of each individual host during the infection. (B) The statuses of the virus during the infection.


\clearpage

{\bf Acknowledgements}


{\bf Author Contributions}

[As per web form]

{\bf Figure Legends}

\clearpage





\end{document}
