\documentclass[12pt,a4paper]{article}
% Always needed
\usepackage{times}
\usepackage{parskip}
\usepackage{graphicx}
\usepackage{setspace}
\usepackage{url}
\usepackage{lscape}
\usepackage{multirow}
\doublespacing
\raggedright
\begin{document}
Classification:

\large {\bf Short-sighted evolution of Influenza cellular receptor binding avidity} \normalsize

Hsiang-Yu Yuan$^{1}$, James Hay$^{1}$, Christophe fraser$^{2}$, Katia Koelle$^{3}$

\begin{tabbing}
$^1$     \= MRC Centre for Outbreak Analysis and Disease Modelling \\
        \> Department of Infectious Disease Epidemiology \\
        \> School of Public Health \\
        \> Imperial College London \\
        \> London, United Kingdom \\ \\
        
$^2$     \= \\

$^3$     \> Department of Biology \\
            \> Duke University \\
            \> Durham, NC United States \\ \\
        

\end{tabbing}
\doublespacing
$^*$Corresponding author: hyuan@imperial.ac.uk, katia.koelle@duke.edu
\clearpage





{\bf Abstract}
Influenza viruses circulating in humans are known to undergo rapid antigenic evolution. Significant work remains, however, in understanding the selective drivers of these evolutionary dynamics and the determining factors of antigenic clusters. Commonly, influenza’s antigenic drift is thought to arise from selection acting directly on drift variants. However, recent studies highlight the importance roles of influenza cellular binding avidity and deleterious mutations in the antigenic drift. \\
Here, we extended this binding avidity hypothesis for influenza’s antigenic drift from serial passaging experiments into population level through a combination of viral sequence analysis and simulation modeling. We first demonstrated that net charge was a good marker of receptor binding avidity and it appeared to be a phenotype under immune-mediated selection in the human infection history. Next, we constructed an individual based model incorporating immune history and within host binding avidity evolution. The model showed that while receptor binding avidity adaptation enhance the survival of virus population, the trait was under stabilizing selection, in which the binding avidity changes within individuals towards either high or low binding avidity was a case of ‘short-sighted’ evolution that decrease a virus’s reproductive potential at the level of the population. Finally, the phylogenetic analysis of influenza H3N2 showed the consistent results that binding avidity was under stabilised selection that the trait of binding avidity was conserved in internal nodes than external tips. Together, we concluded that the binding avidity adaptation within the host produced deleterious effects at the population, which could eventually change the optimum vaccination policy.

{\bf Introduction}

% antigenic drift cause reinfection
Human seasonal influenza viruses reinfected human population each year, one of the main reasons is that the viruses are known to undergo antigenic drift, a process by which amino acid substitutions occurring in epitope regions of the viral hemagglutinin (HA) protein over time, resulting in escape of the immunity induced from previous infection or vaccination. These antigenic changes enable influenza to reinfect previously infected individuals within several years; as a result, seasonal influenza vaccine strains requires constant updating on a close to annual basis (\cite{Carrat2007}). Understanding the factors driving antigenic drift is therefore critical for understanding patterns of disease incidence and for implementing effective influenza planning and control strategies. \\
% antigenic drift mechanism
In addition to the common thought that antigenic drift is a consequence of antibody selection acting on epitope regions (\cite{Grenfell2004} \cite{Wilson1990}), Hensley et al., proposed, inspired from the observation of the serial passaging of the viruses in mice, the possibility that antigenic drift could also be driven by frequent alterations of HA binding avidity (\cite{Hensley2009}). According to the mice experiment, the cellular receptor binding avidity of HA increased while the viruses were passaged to immune mice and on vice versa, decreased while they were passaged to naive mice. Because a large proportion of the binding avidity mutations located in the epitope domain, antigenicity would likely to be altered, as a by product, accompanied with binding avidity mutations. Consequently, antigenic drift could be driven by frequent HA binding avidity changes during disease transmission in a heterogenous population with different levels of immunity. Vaccination targeting on children was proposed to reduce the antigenic drift rate by minimizing the immune heterogeneity in a population. Our previous simulation also confirmed the effect of lowering antigenic drift by targeting on naive individuals (\cite{Yuan2013}). \\
% Current challenges in antigenic drift mechanism
However, challenges exist in projecting the mechanism of binding avidity changes in mice serial passaging experiments onto the disease transmission in the human population. First, despite the binding avidity adaptation in mice, there is currently a lack of population studies that demonstrate influenza viruses binding avidity adaptation to immunity in human. Second, serial passaging studies allow immune host to be infected through the injection with high viral dosage, whereas  the stringency of bottleneck during the natural transmission of the viruses (\cite{Varble2014}) resulted in less likely for viruses transmitted to immune hosts with strong immunity. Third, in the passaging studies, the finely scaled immunity and their corresponding clinical protection, like that existed in a natural human population, are ignored (\cite{Yuan2016}). The heterogeneities in population immunity would affect the fitness of the binding avidity mutations and produce different impacts on disease transmission With immune profile in the population changing over time during the epidemics, the effects of the receptor binding mutations in their epidemiological and evolutionary dynamics would therefore vary accordingly. Furthermore, although higher binding mutations enabled viruses to escape from pre-existing immunity, lower fitness were observed when they were re-passaged into naive mice. This indicates that the mutations with higher binding avidity could possess additional costs or harmful effects while the viruses are replicated in persons with low immunity. Recent studies have demonstrated the mutations having deleterious effects would be important factors that affect the fitness of strains (\cite{Luksza2014}) and their evolution (\cite{Koelle2015}). Together, a more understanding of evolution of the receptor binding avidity in a human population with changing immunity is essential for understanding patterns of disease incidence and for implementing effective influenza vaccination strategies. \\
To understand how virus binding avidity evolves and their impacts on disease transmission, we first demonstrated  binding avidity changes correlated to age-specific antibody prevalence in human population. We also developed an individual-based model linking within-host virus evolution to population level disease transmission, coupled with individual host immune response. The finely scaled immunity and their corresponding clinical protection, like that existed in a natural human population, are included. We demonstrated that short-sighted evolution existed such that the viral lineages with less variabilities would persist longer, indicating the deleterious effects of frequent changes of binding avidity.  Finally, phylogenetic analysis of public influenza virus sequences was performed and supported that binding avidity of human influenza viruses evolves short-sightedly. \\




{\bf Results}

%1. Netcharge is a good marker of binding avidity
Net charge of HA appeared to be a good marker for receptor binding avidity of influenza viruses.  Since there are no large scale systematic studies on binding avidity of different viral mutants, in order to evaluate whether the adaptations of the binding avidity to host immunity can be observed within human influenza infection history, we first investigated whether changes of net charge of HA can be used as molecular markers for binding avidity since net charge is considered to affect receptor binding avidity (\cite{Arinaminpathy2010}). Because the sialic acid receptor on host cells has a negative net charge, such that a more positive net charge of the viral HA is likely to increase the electrostatic force between the viral HA and its sialic acid receptor. We collected the public available amino acid substitutions with corresponding receptor-destroying enzyme (RDE) activities from the literatures and grouped them into groups according to whether the amino acid substitutions reduced (negative), maintained (neutral), or increased (positive) the net charge. Interestingly, the largest increase of binding avidity was among the positive amino acid group and the lowest increase was among the negative group (Fig1).  We also observed that the substitutions which have higher net charge enhanced binding avidity significantly comparing to the mutations having reduced net charge (p value = 0.0001; TableS1), suggesting that net charge appeared to be good markers for HA binding avidity. A significant correlation also occurred when the binding avidity was compared to the absolute net charge using the HA sequences (FigureS1). \\
%2. Netcharge and seroprevalence show similar distributions
The age distribution of net charge supported the presence of binding avidity adaptation in human infections. We collected influenza H3N2 sequences with infected persons’ age metadata isolated from 1994-2005 as part of an Influenza Genome Sequencing Project  [20]. The percentage of the viruses having a high absolute net charge (defined as the absolute net charge > median value: 18) followed a V-shaped distribution by ages. The valley of the distribution occurred in the persons between 40-60 years old (yo) and the two peaks occurred in the age groups 0-20 and 60-80 yo (Fig2). The net charge distribution of different age groups from these human virus isolates was compared to the age specific serological prevalence. Interestingly, the antibody responses by age against the H3N2 strains from 1995 to 2005 also showed the similar V-shaped distribution, with two maximums occurred in young adults and the elderly and the minimum occurred between 40-60 yo (\cite{Kucharski2015a}). The similar pattern was also found in age profile of immunity against H3N2 predicted by a seasonal strain model with different population (\cite{Kucharski2012}). The congruent V-shaped age distribution that were observed from both the net charge of the isolated viruses and the population immunity with the significant correlation between the net charge and binding avidity, supported the hypothesis that a higher binding avidity could be selected by higher host immunity in human population, indicating the presence of binding avidity adaptation in human influenza history and (Hensley 2009).  \\
%3. We constructed a model to study the Effect of Binding Avidity Adaptation 
To understand the impact of influenza binding avidity on epidemic dynamics, we constructed an individual-based epidemic model, coupled with within-host binding avidity adaptation (\cite{Yuan2013a}) and individual antibody responses (ref:stratified immunity). Within each infected individual, the virus binding avidity adapted to the host immunity and reached a higher fitness defined as the within-host reproductive number $R_{in}$ (Fig3a). Different level of host immunity was mapped to the corresponding susceptibility (defined as the probability of infection given a contact) with different virus binding avidity. The A higher host immunity would reduce the probability of infection and would increase the optimum binding avidity to escape immunity (Fig3b). To measure the impact of binding avidity during the epidemics, we first simulated the transient disease dynamics by introducing one antigenic mutant strain into a heterogeneous population with pre-existing partial immunity. \\
%4. Binding Avidity Adaptation prolong the incidence curve 
Our simulation results demonstrated that the within-host binding avidity adaptation prolongs the epidemic period. After the seeding viruses were introduced with a low initial binding avidity value, peak incidence was achieved after approximately 200 days (Fig4). During the outbreak, the increase of the binding avidity was correlated to the increase of average population immunity. The number of naive individuals (defined as pairwise immunity level = 0) dropped as successful infections increased the level of immunity in the population (Fig5).  The number of individuals with higher immunity increased rapidly as the incidence approached the maximum peak, during which a greater amount of antibody boosting would be produced. The average binding avidity was near 0.45 in the early phase (when different initial binding avidities were used, the avidities quickly adapted to near 0.45 before the incidence peak) of the epidemic and later increased to around 0.6 to adapt to the higher immunity population in the late phase. We compare the result to the scenario with binding avidity that was fixed to the early adapted and the late adapted values. The adaptation of binding avidity prolonged the incidence curve until it faded out. A higher amount of the incidence was produced comparing to the viruses whose binding avidity fixed at early phase value. However, interestingly, the maximum incidence was lower than the viruses whose binding avidity fixed at the late adapted value, indicating that the harmful effects of binding avidity adaptation on disease transmission.  \\
%5. Binding avidity adaptation are deleterious
Within-host binding avidity adaptation produced a deleterious effect on disease transmission in the population. To understand why within host binding avidity adaptation produce a lower incidence than the viruses whose binding avidity fixed at the adapted value, we calculated the optimum binding avidity at the population level by time along with the changes in population immunity over the course of the outbreak. The optimum binding avidity is defined as the avidity at which the greatest number of viral offspring can be produced, as measured by the effective reproductive number $R_{t}$. During the outbreak, the average binding avidity among infected individuals before within host adaptation (at the time of infection) was lower than the population optimum binding avidity. Furthermore, the average adapted binding avidity after the within host evolution (at the end of the infection period) was even farther away than the optimum binding avidity and produced an even less $R_{t}$ (Fig6). Interestingly, this demonstrated that within-host binding avidity adaptation appeared to lower the population fitness. \\
%6. Adaptation of Adsorption mutations can be explained by Mutation-Selection Balance phenomenon
Within-host binding avidity evolution was short-sighted in a population with partial immunity. To understand why the binding avidity adapted to a lower fitness, we compared the within host adaptation of viral traits to the population level fitness with different antigenic changes at different stage of the outbreak. Given the distribution of individual immunity, we first calculated the possible evolutionary outcome of a viral mutant. During disease transmission in a population, the chance of the viral mutant being survived depends on the evolutionary fitness from both within-host selection ($R_{in}$) and between-host selection ($R_{t}$). \\
When a strain with a random mutation (could be more than one amino acid site) in binding avidity trait was introduced in a partially protected population in day1, the average number of this mutation being reproduced within the expected infected individuals was approximately a linear curve decreasing monotonically with binding avidity regardless of the magnitude of antigenicity change (Fig7). Without or only with moderate antigenic changes, this distribution contradicted to the between-host selection at the population level, as the population fitness tended to be optimal at a higher binding avidity during the outbreak (Fig7). Thus, although lower binding avidity mutations were more likely to be reproduced within naive individuals, the short-sighted within-host evolution resulted in a deleterious effect at the population level since the probability of infection by viruses with low binding to the persons with partial immunity was low (Fig3b). The accumulated herd immunity through depletion of susceptibles during the epidemics made larger the deleterious effects of within-host binding adaptation (FigSx). When the antigenicity change became strong such that all the individuals became naive, then the low binding avidity mutation would produce a higher fitness on both within-host and population level through the compensatory effect. \\
%7. The Deleterious Effect of Binding Avidity Adaptation on viral phylogeny
The deleterious effect of adaptive binding avidity was further demonstrated by comparing the binding values at the internal and external nodes of the viral phylogeny constructed from our individual based simulation (FigureS2). When binding avidity adaptation was present, many of  the ‘mutants’ which adapted to very low binding avidity were present as external tips. However, the viruses that were present in internal nodes (more likely to produce more offspring in the population), displayed a limited range of binding avidities between average binding from external tips and the optimum binding avidity. This was due to the balance of the selection force from the total population and within-host adaptation as described above. To test the impact the deleterious effect of binding avidity on transmission, we adjust the probability of infecting hosts with different levels of immunity to be less differential and more uniform, as would be the case in a serial passaging experiment without deleterious effects. In this case, the viral phylogeny showed a more homogenous probability distribution of binding avidities at internal and external nodes (FigureS3). This explained the deleterious effects were caused by the differential selection due to the change of the immune profile in the population. \\
%8. Phylogenetic Analysis
The larger variations of net charge present in the external than internal nodes in the reconstructed phylogenetic tree support the short-sighted evolution of binding avidity. To test whether the predicted deleterious effects can actually be seen in influenza H3N2 infection history, we collected the viral sequences isolated from North America and reconstructed the ancestral state of the nucleotide. We observed net charge changes between different clades, but in each clade, we did not observe frequent sequential alterations within the clades (Fig9A). Although the net charge was widely distributed among each year, when the strains were grouped by the number of glycosylation binding sites (which could possibly affect the binding avidity), the net charge maintain a very narrow variance in the trunk and the other internal nodes (Fig9B). For all internal nodes, only 7.7% of the strains showed net charge change; whereas in external tips, which were destined to die out soon, a higher proportion of the strains (12.1%) changed the net chage (TableS3). A lower average netcharge with a higher variance were seen as external tips 17.45 $ \pm1.18$) than internal nodes (17.52 $ \pm1.19$), consistent to the simulated phylogeny from our model (FigureS2). Internal nodes also demonstrated a higher proportion of the high net charge, indicating a better adaptation on partial protected population than external strains. \\




{\bf Discussions}
% Summary
We have demonstrated that influenza net charge is significantly correlated to its binding avidity. A similarity of the fraction of the high net charge viral strains and seroprevalence is observed, indicating the adaptation of influenza binding avidity to herd immunity in human population is likely present. We have developed an individual-based epidemic model incorporated the within-host virus binding avidity adaptation, individual immune boosting and host infection history. The model allows us to produce the simulated viral phylogeny and to observe the changes of virus characteristics (such as antigenic change and binding avidity) along with the changes of the individual immunity in the population by time. The model predicts a short-sighted evolution of binding avidity: First, the the binding avidity after the within-host adaptation would lower the reproductive number than that before the within-host adaptation. Second, the low binding avidity in naive individuals could lower the fitness due to the deleterious effects of low binding in a population with partial protection. Third, the external nodes, which are going extinction, demonstrate a larger net charge variation and a lower net charge. The similar observation was found in the influenza H3N2 phylogenetic tree, which displays a less net charge variation along the internal lineages and a lower net charge in external tips. \\
%Short sight evolution of binding avidity within-host adaptation
We demonstrated that the within host viral binding avidity adaptation would ‘produce’ (select) the deleterious mutations. Based on the observation of within-host adaptation in the serial passaging experiments,  we link the within-host mechanism to human population, in which the differential susceptibility caused by partial immunity would need to be considered. With partial immunity, we didn’t observe the significant alternative changes on net charge happened sequentially among viral lineages in a single season or a clade in our phylogenetic analysis nor from our model simulation, which could continually drive the antigenicity change as a by-product. This is because that binding avidity within-host adaptation is demonstrated to be short-sighted in a heterogenous population, such that the average binding avidity at the end of the infectious period evolves farther away from optimum binding avidity comparing to the binding avidity at the beginning of the infectious period, resulting a further reduction in reproductive number. For example, when influenza virus binding avidity decreases once a naive host is infected, the low binding avidity could become deleterious during the epidemics in a population with partial immunity. The short-sighted within-host evolution has also been predicted from other RNA viruses such as HIV (\cite{Lythgoe2013}). Once within-host competition increases, HIV evolves to higher virulence; however, the fitness of the viral population at the epidemiological level decreases. For influenza, the more viruses adapt to either naive individuals or that with strong immunity, the less fitness would be produced in a  heterogeneous population. \\
%Antigenic drift could be driven by binding avidity change as Compensatory mutation 
Antigenic change with compensatory binding avidity mutation would cause the main antigenic drift. Based on the observation of within-host adaptation in the serial passaging experiments,  we have extended the within-host adaptation mechanism from the serial passaging experiment results to human population, in which the differential susceptibility is considered. We haven’t observed the alternative and sequential changes of net charge occurred significantly among viral lineages, which could continually drive the antigenicity drift as a by-product, from our model simulation nor from the reconstructed phylogenetic tree. We have observed that the virus binding avidity maintains in a more stabilized level in internal nodes than external tips with the seasons. While the average avidity between the seasons changes, the increase of the avidity would be a result from adaptation to an increased herd immunity and the decrease of that could be caused by the compensatory binding mutations accompanied with large scale antigenic changes. When an antigenic mutation occurs with large effect, optimum binding avidity would produce a higher fitness through compensatory effects, which can explain that the major substitutions in influenza history are located in receptor binding sites (\cite{Burke2013}).  \\
%Binding avidity prolong the length of the incidence 
This study raise the fundamental question regarding the effects of binding avidity adaptation in influenza evolution: why influenza viruses adopt the strategy that to lengthen the survival time of the virus lineages to wait for a new antigenic change than simply to increase the rate of incidence? Our results are consistent to several assumptions that have been made to support short-sighted hypothesis including within-host evolution towards a higher virulence or a to escape from the immune system’s recognition (\cite{Levin1994}). However, here we also demonstrated that the binding avidity evolution, even though short-sighted, could potentially reduce the chance of virus extinction. \\
%Future challenges
Due to the short-sighted evolution, the optimum vaccination strategy becomes more difficult than it was believed. A possible vaccination strategy that would reduce the transmission events  in a heterogenous population is to immunize the persons with only partial protection. In a population with partial herd immunity, ,multiple antigenic beneficial strains would likely co-exist together before the final antigenic strain replaces others and become dominant. Among them, the one with optimum compensatory mutations on binding avidity that would increase transmissibility most is likely to become the progenitor of the future antigenic strain. Therefore, the study highlights the need of incorporating binding avidity change in addition to serological titres to select or predict possible future vaccine strain. To understand the effect of binding avidity on disease dynamics and transmissibility and the evolutionary strategy of binding avidity of the viruses, would be beneficial for future antigenic strain prediction. \\




{\bf Material \& Methods}

\textit{Analysis of HA cellular receptor binding avidity} \\
The RDE binding avidity for each single amino acid changes were both collected 34 mutants from the publicly available literatures (\cite{Hensley2009} \cite{Das2011} \cite{Myers2013} \cite{Li}) and given by the group performing the RDE assay for monoclonal antibodies selected sites (\cite{Hensley2009}). Totally there were 74 mutants collected, including 53 single, 20 double and 1 triple amino acid changes. For the RDE binding avidity assay present in literatures, the magnitudes were extracted using the Web Plot Digitizer(\cite{Rohatgi2012}). The effects of each single amino acid changes on binding avidity were calculated as the log ratio of RDE activity of the mutant to the wild type (TableS2).

\textit{Calculating net charge of influenza HA} \\
The influenza virus sequence isolated from New York State as part of an Influenza Genome Sequencing Project (\cite{Ghedin2005}) along with the exact date of isolation and clinical metadata on host age were collected from the Influenza Virus Resource Database (\cite{Bao2008}). We obtained in total 686 full-length influenza A/H3N2 HA sequences isolated between 1993 and 2005. Changes of the net charge were estimated by summing the total number of positive and subtracting the negative  charged amino acids. Amino acid such as Arginine (R), Histidine (H) and Lysine (K) are positively charged while Aspartic acid (E) and Glutamic acid (D) are negatively charged. For estimating net charge from the isolated viral HA sequences, net charge was calculated from the sequences using amino acid positions 17 through 345 of the HA, which defined the protein’s globular head domain (HA1). We specifically did not calculate net charge values using only the amino acid positions that comprise the cellular receptor binding site because charge at other sites in the globular head domain also impact electrostatic forces and thereby binding avidity. The calculation involved subtracting the total number of negatively charged amino acid residues from the total number of positively charged amino acid residues.

\textit{Phylogenetic Analysis} \\
Phylogenetic trees were reconstructed using the software program BEAST (\cite{Drummond2012}), under a general time reversible (GTR) model with gamma distributed rate variation and a proportion of invariant sites. 107 MCMC steps were sufficient in reaching a high effective sample size. From each viral subtype’s maximum clade credibility (MCC) tree, we inferred the ancestral states of the observed viral sequences. If BEAST produced multiple possible ancestor sequences, consensus sequences were calculated with BLOSUM50 scoring matrix using Matlab Bioinformatics Toolbox. For all the observed external and inferred internal sequences we calculated net charge as described above.

\textit{The Individual based modelling} \\
We created an individual based model, in which each individual host was represented as the statuses: susceptible ($S$), infected ($I$) and recovered ($R$), respectively with the finely scaled life course immunity $K$ (FigureS4A). The life course immunity was defined as the immunity against the first infected strain or other strains with the same antigenicity. We consider each virus as an unique viral strain replicating in the infected host and possessing certain phenotypic characteristics such as the antigenic change $\delta$ and the binding avidity $V$. The infection histories $h$ of the hosts were able to be tracked by specifying the set of viral strains to which each individual host was exposed to. Since each viral strain underwent antigenicity changes, we also defined a pairwise immunity $J$ as the host immunity, given the infection history, against the challenging virus $v$.

The model was simulated under demographic stochasticity using Gillespie’s tau-leap algorithm (\cite{Gillespie2001}) to calculate the probability of number of events for each individual in an unit time interval. We assumed that births of new hosts, starting from no life course immunity ($K = 0$), occurred at a rate 1/70 per year. Similarly, deaths occurred also at the same rate for the entire population. Total number of contacts was calculated based on the contact rate c, number of susceptibles X, and number of infecteds Y in a total population N, resulting in the total number of contacts cXY/N in a unit time. As shown in FigureS4A and B, for each contact, a random susceptible target $S_{tar}$ and an infected source $I_{src}$ were drawn from the population, and the probability of successful transmission, $\rho$, was determined by the pairwise immunity $J$ of the target host against the challenging source virus $v_{i}$, and the binding avidity V of the virus at the contact time (FigureS4B). The pairwise immunity , $J =K - min(\delta_{ih})$, was the life course immunity that subtracted the minimum antigenic distance between the current challenging virus $v_{i}$ and the previous infected strains in the infection history. Once the individual became infected, the average infectious period $1/\gamma$ would lasted 3.3 days, during which the virus’ binding avidity $V$ was adapting to the host immunity $J$ determined by the fitness gradient (\cite{Yuan2013}) and producing the antigenic changes $\delta$.

After the infected individual was recovered, the individual became transiently fully protected within 25 days ($1/\omega$) on average by short-lived immunity, possibly mediated by cytotoxic T lymphocytes with other immunological factors ( [39] [40]). During the recovered period, individual life course immunity $k$ was boosted following a Poisson distribution with mean boosting level $g=3$, which produce nearly $70\%$ clinical protection during primary infection, similar to our previous estimation (\cite{Yuan2016}). After the transient immunity waned, the recovered individual became susceptible again and were protected from viral infection by the increased antibodies pairwise immunity $j$.

\textit{Probability of infection} \\
Each time when a virus $v$ was contacted to a susceptible host $S$, the probability of successfully infecting that host was defined as the likelihood of the viruses not going stochastic extinction in a viral population within that host:

$\rho = \frac{1}{1-(R_{in})^\sigma}$

where $R_{in}$ is the reproductive number in a host and $\sigma$ is the number of virions initially transmitted. $R_{in}$ was defined as the expected number of viruses which can be produced by a single virus in a host with immunity J.

$R_{in}=n \cdot f(J,V) \cdot g(V)$

Where $f(J,V) = [1 - e^{p(V+1)}]^r$ and $J$ is the probability of evading the immune response with a given virus binding avidity $V$, and a given host immunity $J$ ;  and $p$ and $r$ are scaling constants. $g(V) = e^{(aV^b)}$is the probability of successful replication within a host with a given binding avidity $V$, where $a$ and $b$ are scaling constants (\cite{Yuan2013}). Combining these probabilities with the average number of offspring $n$ produced after the replication by a single virus give the reproductive number for viruses within a host.

\textit{Pairwise immunity and immune history} \\
We considered host pairwise immunity as having partial protection against different viruses depending on the nearest antigenic distance between the infecting virus and the viruses within the host’s immune history. The pairwise immunity of the target host $t$ against the challenging virus $v_{i}$ in the infected source, was given by $J_{mi}=K_{m}-min(D_{i\hbar_{m}})$, where $K_{m}$ is the life course immunity of host $m$ against the primary infecting strain or the same strains without antigenicity changes, $D_{i\hbar_{m}}$ is the set of the antigenic distances between the infecting virus i and the set of infected viruses among infection history $\hbar$ in the host $m$. Following a contact, we find the shortest antigenic distance $min(D_{i\hbar_{m}})$ between the current challenging virus and the previous viral strains in host’s infection history by tracing back along the transmission tree to find the last common ancestor of the infecting virus $v_{i}$ and the infection history viruses $\hbar_{m}$ and calculate the sum of each individual antigenic changes along the path.

\textit{Within host adaptation and antigenicity change} \\
During infectious period, binding avidity $V$ of each infected strain adapted to the host immunity $J$. The change of binding avidity by time $\Delta V(t)$ is the product of the rate of change of the viral binding avidity $dv/dt =V_{c}(dR_{in}/dv)$ and the time elapse $\Delta t$ (\cite{Yuan2013}), where $V$ is the within-host binding avidity of the virus, $dR_{in}/dV$ is the fitness gradient, and $V_{c}=0.075$. We considered the scenario where antigenic drift occurred as sum of the epitope changes that contains both randomly produced antigenic mutations, which comply with the classical hypothesis of antibody selected antigenicity changes, and the mutations produced as the by products of binding avidity changes. Random antigenic mutations occurred with a probability $p$, and the effect of the antigenic change $r$ was drawn from an exponential distribution with average antigenic distance $d=0.1$ could occur. Changes in binding avidity also translated into small changes in antigenicity described by parameter $\kappa$, such that rate of antigenic change due to binding avidity change was proportional to the rate of change of binding avidity. Therefore the total antigenic change happened in one virus particle was expected to be

$\Delta A=rp+\kappa \Delta V.$

\textit{Initial population immunity} \\
We began each simulation using initial conditions of the immune profile of susceptible individuals that generated from a single epidemic peak. To match the distribution of partial immunity that were observed in population, we ran the simulation from a completely naive population to generate a distribution of host immunities. The epidemics generated a total incidence of roughly 50\%. Each infected individual received a boost to immunity drawn from a Poisson distribution with mean $mu=6$, resulting in a distribution of host immunities, $K$. All simulations were then run using the same host immunity distribution, $S_{k}$, introducing a new virus with antigenic distance $\Delta A=2$ from the host’s immunity. The resulting distribution of $J$ was overall similar to serological data for seasonal influenza (\cite{Yuan2016}).

\textit{The fitness in a changing immune profile environment} \\
The effective reproductive number was calculated while the total number of and the immune profile composition of susceptible individuals changed by time. Given a single virus with binding avidity V, the reproductive number was:

\begin{equation}
R=\frac{\sum_{J=0}^{\infty}\beta(J,V)S_{t}(J)}{\gamma N}
\end{equation}

The average within host relative fitness $\bar{w}$ was calculated from the the immune profile composition of susceptible individuals. We first calculated the newly infected individuals with different immunity, because all the mutants are replicated only in the infected hosts. The number of newly infected individuals is
\begin{equation}
I_{tn}(J) = \beta(J,V)\frac{S_{t}(J)}{N} {I_{t}}
\end{equation}
Under the assumption that the fixation of the mutation is proportional to its within-host fitness in infected individuals (\cite{Gillespie1984}). The the average within host relative fitness $w$ is
\begin{equation}
\bar{w}_{t}(V)=\frac{\sum_{J=0}^{\infty} R_{in}(J,V)I_{tn}^*(J,V)}{\max{R_{in}(J,V)}}
\end{equation}
where $I_{tn}^*(J,V)$ is normalized to be $\frac{I_{tn}(J,V)}{\sum_{J=0}^{\infty}I_{tn}(J,V)}$ to consider the scenario that every initial binding avidity are equally distributed.

\clearpage

{\bf Figure Legends} \\
Figure1. The impact on binding avidity of single amino acid mutations by net charge. The value is calculated as the log ratio of RDE activity of the mutant to the wild type. Positive represents the amino acid change that increase net charge by 1 or 2 units. Negative represents the amino acid change that decrease net charge by 1 or 2 units. Neutral represents net charge is same. Charged amino acids include aspartate (D), glutamate (E) are counted as negative (-1) and Histidine (H), Arginine (R), Lysine (K) for positive (+1) are counted as positive (+1).
 
Figure2. Net charge distribution by age group. The proportion of higher net charge is calculated for 5 age groups. The viruses sequences and clinical data are obtained from the Influenza Genome Sequencing Project (\cite{Ghedin2005}) and the Influenza Virus Resource Database (\cite{Bao2008}). 

Figure3. Within host fitness and the probability of infection. (A) Within-host fitness represented by the expected reproductive number $R_{in}$. Parameters for within-host immune escape and replication cost, p=4, r=70, a=0.7 and b=3, and parameter for virus offspring number $n=4$. Bluer colours indicate much naive individuals where redder colours indicate much immune individuals. (B) The probability of infection with differential selection. Within-host parameters values are same as (A) and the effective initial number of the transmitted virus $\sigma$=1. (C) The probability of infection with weak differential selection. Parameters values same as (A) but the effective initial number of the transmitted virus v=3 to resembles higher viral dosages in serial passaging transmission.

Figure4. Incidence over time for the various binding avidity scenarios. The blue line shows incidence when binding avidity is fixed near the late stage population average of V=0.6, eliciting a rapid, large peak. The green line shows the incidence when binding avidity is fixed near the early phase population average at V=0.45, resulting in a very small epidemic peak. The red line shows the incidence when binding avidity is allowed to adapt within hosts over the course of the epidemic, starting at a binding avidity of V=0.45. When binding avidity is allowed to adapt within the host, the overall incidence is higher than in the suboptimal binding avidity case, but lower than the scenario with fixed late-phase average binding avidity. In all scenarios, Total population N is set to be 1,000,000. For the initial day, the number of seed viruses I0 = 100, recovered hosts R0 = 0 and the remaining hosts were susceptible under the initial immunity distribution as described in Methods. Contact rate, c=0.7, birth/death rate, m=1/(70*365)(days-1), waning rate, w = 1/25, recovery rate, g = 1/3.3 (days-1), genetic variance, Vc=0.075, mean boost to immunity, mu = 6, initial antigenic distance, delta=2. Within host parameters were as described in FigureS1. Initial conditions were generated as described in the Methods.

Figure5. Distribution of susceptible host immunity over time for the adaptive binding avidity scenario as described in Figure4. Colours show the proportion of susceptible hosts with a given level of effective immunity against the seed virus, J, over time. The white line represents the population mean immunity level. At the start of the simulation, the majority of hosts are completely naive to the novel strain, with some hosts exhibiting low levels of immunity due to prior infection. As incidence increases, the mean host immunity also increases as infected hosts recover and develop long-term immunity. This coincides with the peak of incidence at around 200 days. Parameters are as described in Figure4 and FigureS1.

Figure6. The effective reproductive number changes by time and binding avidity. The effective reproductive number was calculated assuming the population contains only susceptible individuals with different protection (as Figure5) but excluding the transient recovered individuals and infected individuals. The optimum binding (dotted line) produces the largest number of offsprings. Red, the mean starting binding avidity during infectious period. Light blue, the mean final binding avidity during infectious period. Bold gray, the optimum reproductive number. Thin gray, the optim reproductive number when binding avidity is 0.

Figure7. Within-host and between-host relative fitness by different binding avidity and antigenic  changes. Within-host relative fitness is shown in red surface. Between-host relative fitness is shown in a surface from blue (low) to yellow (high).

Figure8. Binding avidity changes from the sampled viral phylogeny. Blue dots, internal nodes from the viral phylogeny. Green dots, external tips from the viral phylogeny. Dotted line was the optimum virus binding avidity that will produce largest offsprings. Bold gray, mean binding avidity from internal nodes.

Figure9. Cellular receptor binding avidity dynamics analyzed phylogenetically. (A) The maximum clade credibility (MCC) tree for the H3N2 viral isolates from New York State, with HA net charge values mapped onto all inferred internal nodes and all observed external nodes. Numbers next to the bracketed clades denote the $\#NGS$ group. (B) Viral net charge dynamics over time. Each point represents a node from (A) and is colored by whether it is classified as an internal trunk node (red circle), an internal non-trunk node (blue dot), or an external node (green ‘x’). Its placement along the x-axis corresponds to the inferred or observed time of the node in the phylogeny. Its placement along the y-axis corresponds to its calculated net charge value. Each of the four subplots shows viral net charge dynamics for a single $\#NGS$ group. The total number of net charge changes occurring along internal branches (including trunk and non-trunk internal branches) was approximately $40\%$ less than the total number of net charge changes occurring along external branches. Normalizing by total branch lengths of internal and external branches yielded similar results, with fewer number of net charge changes occurring per unit time on internal relative to external branches.
\clearpage




{\bf Supplementary}
\clearpage

\begin{table}
{\bf Table S1.} Comparison of the net charge properties in both internal and external nodes from the reconstructed phylogenetic tree.

\begin{tabular*}{10cm}{rrrrrr}
\hline\hline \\%inserts double horizontal lines


Net charge&  Dec. or Neu. &  Inc. & Total  & \\
\hline \\ %inserts double horizontal lines
$ \geq 0$ &  39           &  10   & 49     &  \\
$ < 0$ &   6           &  19   & 25     &  \\
Total  &  45           &  29   & 74   &  \\
\hline %inserts double horizontal lines
\end{tabular*}
\end{table}
P value = 0.0001 using Fisher's exact test

\clearpage


\begin{table}
{\bf Table S2.} The comparison of the seroprevalence observed sera and the titre model output. The seropravalene (\%) among each age group during the baseline and the follow-up rounds are listed.

\begin{tabular*}{16cm}{rrrrrrr}

\hline\hline \\%inserts double horizontal lines

  & Total strains&  \parbox[c]{1.8 cm}{\raggedright No. strains change netcharge}  &    \parbox[c]{1.8cm}{\raggedright Positive changes} &   \parbox[c]{1.8cm}{\raggedright Negative changes} &     \parbox[c]{1.8cm}{\raggedright Prop High Netcharge}    &     \parbox[c]{1.8cm}{\raggedright Avg netcharge}   \\
\hline \\ %inserts double horizontal lines
Internal &  684 & 53 (7.7\%) &  15 &   38 & 20.0 (\%) & 17.52 $\pm 1.13$\\ \\
External &  686 & 83 (12.1\%) &  32 &   51 & 18.7 (\%)& 17.45 $\pm 1.18$\\ \\
\hline %inserts double horizontal lines
\end{tabular*}
\end{table}
\clearpage


{\bf Table S3.} The list of all the collected mutations with the corresponding binding avidity (RDE). (Attached as the excel file)
\clearpage



FigureS1. The relative binding avidity by absolute net charge. The value is calculated as the log ratio of RDE activity of the mutant to the wild type. Absolute net charges of all the mutant strains are calculated. Regression is performed. P-value = x.xx.

FigureS2. Comparison of within-host and between-relative fitness by time during the epidemics.
Red line represents the average relative fitness within newly infected hosts. Blue line represents effective reproductive number in the human population.

FigureS3. Viral phylogeny with weak differential selection on binding avidity. Parameters setting are same as Figure S2A with contact rate c=0.3.

FigureS4. Schema of the disease transmission in the individual-based model. (A) The statuses of each individual host during the infection. (B) The statuses of the virus during the infection.
\clearpage
{\bf Acknowledgements}
{\bf Author Contributions}
[As per web form]
\clearpage
\end{document}



